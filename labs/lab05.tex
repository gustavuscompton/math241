% Options for packages loaded elsewhere
\PassOptionsToPackage{unicode}{hyperref}
\PassOptionsToPackage{hyphens}{url}
\PassOptionsToPackage{dvipsnames,svgnames,x11names}{xcolor}
%
\documentclass[
]{article}
\usepackage{amsmath,amssymb}
\usepackage{iftex}
\ifPDFTeX
  \usepackage[T1]{fontenc}
  \usepackage[utf8]{inputenc}
  \usepackage{textcomp} % provide euro and other symbols
\else % if luatex or xetex
  \usepackage{unicode-math} % this also loads fontspec
  \defaultfontfeatures{Scale=MatchLowercase}
  \defaultfontfeatures[\rmfamily]{Ligatures=TeX,Scale=1}
\fi
\usepackage{lmodern}
\ifPDFTeX\else
  % xetex/luatex font selection
\fi
% Use upquote if available, for straight quotes in verbatim environments
\IfFileExists{upquote.sty}{\usepackage{upquote}}{}
\IfFileExists{microtype.sty}{% use microtype if available
  \usepackage[]{microtype}
  \UseMicrotypeSet[protrusion]{basicmath} % disable protrusion for tt fonts
}{}
\makeatletter
\@ifundefined{KOMAClassName}{% if non-KOMA class
  \IfFileExists{parskip.sty}{%
    \usepackage{parskip}
  }{% else
    \setlength{\parindent}{0pt}
    \setlength{\parskip}{6pt plus 2pt minus 1pt}}
}{% if KOMA class
  \KOMAoptions{parskip=half}}
\makeatother
\usepackage{xcolor}
\usepackage[margin=1in]{geometry}
\usepackage{color}
\usepackage{fancyvrb}
\newcommand{\VerbBar}{|}
\newcommand{\VERB}{\Verb[commandchars=\\\{\}]}
\DefineVerbatimEnvironment{Highlighting}{Verbatim}{commandchars=\\\{\}}
% Add ',fontsize=\small' for more characters per line
\usepackage{framed}
\definecolor{shadecolor}{RGB}{248,248,248}
\newenvironment{Shaded}{\begin{snugshade}}{\end{snugshade}}
\newcommand{\AlertTok}[1]{\textcolor[rgb]{0.94,0.16,0.16}{#1}}
\newcommand{\AnnotationTok}[1]{\textcolor[rgb]{0.56,0.35,0.01}{\textbf{\textit{#1}}}}
\newcommand{\AttributeTok}[1]{\textcolor[rgb]{0.13,0.29,0.53}{#1}}
\newcommand{\BaseNTok}[1]{\textcolor[rgb]{0.00,0.00,0.81}{#1}}
\newcommand{\BuiltInTok}[1]{#1}
\newcommand{\CharTok}[1]{\textcolor[rgb]{0.31,0.60,0.02}{#1}}
\newcommand{\CommentTok}[1]{\textcolor[rgb]{0.56,0.35,0.01}{\textit{#1}}}
\newcommand{\CommentVarTok}[1]{\textcolor[rgb]{0.56,0.35,0.01}{\textbf{\textit{#1}}}}
\newcommand{\ConstantTok}[1]{\textcolor[rgb]{0.56,0.35,0.01}{#1}}
\newcommand{\ControlFlowTok}[1]{\textcolor[rgb]{0.13,0.29,0.53}{\textbf{#1}}}
\newcommand{\DataTypeTok}[1]{\textcolor[rgb]{0.13,0.29,0.53}{#1}}
\newcommand{\DecValTok}[1]{\textcolor[rgb]{0.00,0.00,0.81}{#1}}
\newcommand{\DocumentationTok}[1]{\textcolor[rgb]{0.56,0.35,0.01}{\textbf{\textit{#1}}}}
\newcommand{\ErrorTok}[1]{\textcolor[rgb]{0.64,0.00,0.00}{\textbf{#1}}}
\newcommand{\ExtensionTok}[1]{#1}
\newcommand{\FloatTok}[1]{\textcolor[rgb]{0.00,0.00,0.81}{#1}}
\newcommand{\FunctionTok}[1]{\textcolor[rgb]{0.13,0.29,0.53}{\textbf{#1}}}
\newcommand{\ImportTok}[1]{#1}
\newcommand{\InformationTok}[1]{\textcolor[rgb]{0.56,0.35,0.01}{\textbf{\textit{#1}}}}
\newcommand{\KeywordTok}[1]{\textcolor[rgb]{0.13,0.29,0.53}{\textbf{#1}}}
\newcommand{\NormalTok}[1]{#1}
\newcommand{\OperatorTok}[1]{\textcolor[rgb]{0.81,0.36,0.00}{\textbf{#1}}}
\newcommand{\OtherTok}[1]{\textcolor[rgb]{0.56,0.35,0.01}{#1}}
\newcommand{\PreprocessorTok}[1]{\textcolor[rgb]{0.56,0.35,0.01}{\textit{#1}}}
\newcommand{\RegionMarkerTok}[1]{#1}
\newcommand{\SpecialCharTok}[1]{\textcolor[rgb]{0.81,0.36,0.00}{\textbf{#1}}}
\newcommand{\SpecialStringTok}[1]{\textcolor[rgb]{0.31,0.60,0.02}{#1}}
\newcommand{\StringTok}[1]{\textcolor[rgb]{0.31,0.60,0.02}{#1}}
\newcommand{\VariableTok}[1]{\textcolor[rgb]{0.00,0.00,0.00}{#1}}
\newcommand{\VerbatimStringTok}[1]{\textcolor[rgb]{0.31,0.60,0.02}{#1}}
\newcommand{\WarningTok}[1]{\textcolor[rgb]{0.56,0.35,0.01}{\textbf{\textit{#1}}}}
\usepackage{graphicx}
\makeatletter
\def\maxwidth{\ifdim\Gin@nat@width>\linewidth\linewidth\else\Gin@nat@width\fi}
\def\maxheight{\ifdim\Gin@nat@height>\textheight\textheight\else\Gin@nat@height\fi}
\makeatother
% Scale images if necessary, so that they will not overflow the page
% margins by default, and it is still possible to overwrite the defaults
% using explicit options in \includegraphics[width, height, ...]{}
\setkeys{Gin}{width=\maxwidth,height=\maxheight,keepaspectratio}
% Set default figure placement to htbp
\makeatletter
\def\fps@figure{htbp}
\makeatother
\usepackage{soul}
\setlength{\emergencystretch}{3em} % prevent overfull lines
\providecommand{\tightlist}{%
  \setlength{\itemsep}{0pt}\setlength{\parskip}{0pt}}
\setcounter{secnumdepth}{-\maxdimen} % remove section numbering
\ifLuaTeX
  \usepackage{selnolig}  % disable illegal ligatures
\fi
\IfFileExists{bookmark.sty}{\usepackage{bookmark}}{\usepackage{hyperref}}
\IfFileExists{xurl.sty}{\usepackage{xurl}}{} % add URL line breaks if available
\urlstyle{same}
\hypersetup{
  pdftitle={Lab 5},
  colorlinks=true,
  linkcolor={Maroon},
  filecolor={Maroon},
  citecolor={Blue},
  urlcolor={blue},
  pdfcreator={LaTeX via pandoc}}

\title{Lab 5}
\author{}
\date{\vspace{-2.5em}Math 241, Week 6}

\begin{document}
\maketitle

\begin{Shaded}
\begin{Highlighting}[]
\CommentTok{\# Put all necessary libraries here}
\FunctionTok{library}\NormalTok{(tidyverse)}
\FunctionTok{library}\NormalTok{(rnoaa)}
\FunctionTok{library}\NormalTok{(rvest)}
\FunctionTok{library}\NormalTok{(httr)}
\end{Highlighting}
\end{Shaded}

\hypertarget{due-friday-march-1st-at-830am}{%
\subsection{Due: Friday, March 1st at
8:30am}\label{due-friday-march-1st-at-830am}}

\hypertarget{goals-of-this-lab}{%
\subsection{Goals of this lab}\label{goals-of-this-lab}}

\begin{enumerate}
\def\labelenumi{\arabic{enumi}.}
\tightlist
\item
  Practice grabbing data from the internet.
\item
  Learn to navigate new R packages.
\item
  Grab data from an API (either directly or using an API wrapper).
\item
  Scrape data from the web.
\end{enumerate}

\hypertarget{potential-api-wrapper-packages}{%
\subsection{Potential API Wrapper
Packages}\label{potential-api-wrapper-packages}}

\hypertarget{problem-1-predicting-the-unpredictable-portland-weather}{%
\subsection{\texorpdfstring{Problem 1: Predicting the
\st{Un}predictable: Portland
Weather}{Problem 1: Predicting the Unpredictable: Portland Weather}}\label{problem-1-predicting-the-unpredictable-portland-weather}}

In this problem let's get comfortable with extracting data from the
National Oceanic and Atmospheric Administration's (NOAA) API via the R
API wrapper package \texttt{rnoaa}.

You can find more information about the datasets and variables
\href{https://www.ncdc.noaa.gov/homr/reports}{here}.

\begin{enumerate}
\def\labelenumi{\alph{enumi}.}
\item
  First things first, go to
  \href{https://www.ncdc.noaa.gov/cdo-web/token}{this NOAA website} to
  get a key emailed to you. Then insert your key below:
\item
  From the National Climate Data Center (NCDC) data, use the following
  code to grab the stations in Multnomah County. How many stations are
  in Multnomah County?
\end{enumerate}

\begin{Shaded}
\begin{Highlighting}[]
\NormalTok{stations }\OtherTok{\textless{}{-}} \FunctionTok{ncdc\_stations}\NormalTok{(}\AttributeTok{datasetid =} \StringTok{"GHCND"}\NormalTok{, }
                          \AttributeTok{locationid =} \StringTok{"FIPS:41051"}\NormalTok{)}

\NormalTok{mult\_stations }\OtherTok{\textless{}{-}}\NormalTok{ stations}\SpecialCharTok{$}\NormalTok{data}

\NormalTok{mult\_stations }\SpecialCharTok{\%\textgreater{}\%}
  \FunctionTok{summarize}\NormalTok{(}\AttributeTok{multnomahstations =} \FunctionTok{n}\NormalTok{())}
\end{Highlighting}
\end{Shaded}

\begin{verbatim}
##   multnomahstations
## 1                25
\end{verbatim}

\begin{enumerate}
\def\labelenumi{\alph{enumi}.}
\setcounter{enumi}{2}
\tightlist
\item
  January was not so rainy this year, was it? Let's grab the
  precipitation data for site \texttt{GHCND:US1ORMT0006} for this past
  January.
\end{enumerate}

\begin{Shaded}
\begin{Highlighting}[]
\CommentTok{\# First fill{-}in and run to following to determine the}
\CommentTok{\# datatypeid}
\FunctionTok{ncdc\_datatypes}\NormalTok{(}\AttributeTok{datasetid =} \StringTok{"GHCND"}\NormalTok{,}
               \AttributeTok{stationid =} \StringTok{"GHCND:US1ORMT0006"}\NormalTok{)}
\end{Highlighting}
\end{Shaded}

\begin{verbatim}
## $meta
##   offset count limit
## 1      1     5    25
## 
## $data
##      mindate    maxdate                                   name datacoverage
## 1 1750-02-01 2024-02-27                          Precipitation            1
## 2 1840-05-01 2024-02-27                               Snowfall            1
## 3 1857-01-18 2024-02-27                             Snow depth            1
## 4 1952-07-01 2024-02-27 Water equivalent of snow on the ground            1
## 5 1998-06-01 2024-02-27           Water equivalent of snowfall            1
##     id
## 1 PRCP
## 2 SNOW
## 3 SNWD
## 4 WESD
## 5 WESF
## 
## attr(,"class")
## [1] "ncdc_datatypes"
\end{verbatim}

\begin{Shaded}
\begin{Highlighting}[]
\CommentTok{\# Now grab the data using ncdc()}
\NormalTok{precip\_se\_pdx }\OtherTok{\textless{}{-}} \FunctionTok{ncdc}\NormalTok{(}\AttributeTok{datasetid =} \StringTok{"GHCND"}\NormalTok{,}
               \AttributeTok{stationid =} \StringTok{"GHCND:US1ORMT0006"}\NormalTok{,}
               \AttributeTok{startdate =} \StringTok{"2024{-}01{-}01"}\NormalTok{,}
               \AttributeTok{enddate =} \StringTok{"2024{-}02{-}01"}\NormalTok{)}
\end{Highlighting}
\end{Shaded}

\begin{enumerate}
\def\labelenumi{\alph{enumi}.}
\setcounter{enumi}{3}
\tightlist
\item
  What is the class of \texttt{precip\_se\_dpx}? Grab the data frame
  nested in \texttt{precip\_se\_dpx} and call it
  \texttt{precip\_se\_dpx\_data}.
\end{enumerate}

\begin{Shaded}
\begin{Highlighting}[]
\FunctionTok{class}\NormalTok{(precip\_se\_pdx)}
\end{Highlighting}
\end{Shaded}

\begin{verbatim}
## [1] "ncdc_data"
\end{verbatim}

\begin{Shaded}
\begin{Highlighting}[]
\NormalTok{precip\_se\_pdx\_data }\OtherTok{\textless{}{-}}\NormalTok{ precip\_se\_pdx}\SpecialCharTok{$}\NormalTok{data}
\end{Highlighting}
\end{Shaded}

\begin{enumerate}
\def\labelenumi{\alph{enumi}.}
\setcounter{enumi}{4}
\tightlist
\item
  Use \texttt{ymd\_hms()} in the package \texttt{lubridate} to wrangle
  the date column into the correct format.
\end{enumerate}

\begin{Shaded}
\begin{Highlighting}[]
\NormalTok{precip\_se\_pdx\_data}\SpecialCharTok{$}\NormalTok{date }\OtherTok{\textless{}{-}} \FunctionTok{ymd\_hms}\NormalTok{(precip\_se\_pdx\_data}\SpecialCharTok{$}\NormalTok{date)}
\end{Highlighting}
\end{Shaded}

\begin{enumerate}
\def\labelenumi{\alph{enumi}.}
\setcounter{enumi}{5}
\tightlist
\item
  Plot the precipitation data for this site in Portland over time. Rumor
  has it that we had only one day where it didn't rain. Is that true?
\end{enumerate}

\begin{Shaded}
\begin{Highlighting}[]
\NormalTok{precip\_se\_pdx\_data }\SpecialCharTok{\%\textgreater{}\%}
  \FunctionTok{ggplot}\NormalTok{(}\FunctionTok{aes}\NormalTok{(}\AttributeTok{x =}\NormalTok{ date, }\AttributeTok{y =}\NormalTok{ value, }\AttributeTok{fill =}\NormalTok{ datatype)) }\SpecialCharTok{+}
  \FunctionTok{geom\_col}\NormalTok{()}
\end{Highlighting}
\end{Shaded}

\includegraphics{lab05_files/figure-latex/unnamed-chunk-8-1.pdf} Not
true. This dataset only includes up to January 13th but even with that
missing data we can see that it did not rain for several days at the
beginning of the month.

\begin{enumerate}
\def\labelenumi{\alph{enumi}.}
\setcounter{enumi}{6}
\tightlist
\item
  (Bonus) Adapt the code to create a visualization that compares the
  precipitation data for January over the the last four years. Do you
  notice any trend over time?
\end{enumerate}

\begin{Shaded}
\begin{Highlighting}[]
\CommentTok{\#2020}
\NormalTok{precip\_se\_pdx\_2020 }\OtherTok{\textless{}{-}} \FunctionTok{ncdc}\NormalTok{(}\AttributeTok{datasetid =} \StringTok{"GHCND"}\NormalTok{,}
               \AttributeTok{stationid =} \StringTok{"GHCND:US1ORMT0006"}\NormalTok{,}
               \AttributeTok{startdate =} \StringTok{"2020{-}01{-}01"}\NormalTok{,}
               \AttributeTok{enddate =} \StringTok{"2020{-}02{-}01"}\NormalTok{)}

\NormalTok{precip\_se\_pdx\_2020\_data }\OtherTok{\textless{}{-}}\NormalTok{ precip\_se\_pdx\_2020}\SpecialCharTok{$}\NormalTok{data}

\NormalTok{precip\_se\_pdx\_2020\_data}\SpecialCharTok{$}\NormalTok{date }\OtherTok{\textless{}{-}} \FunctionTok{ymd\_hms}\NormalTok{(precip\_se\_pdx\_2020\_data}\SpecialCharTok{$}\NormalTok{date)}
\CommentTok{\#2021}
\NormalTok{precip\_se\_pdx\_2021 }\OtherTok{\textless{}{-}} \FunctionTok{ncdc}\NormalTok{(}\AttributeTok{datasetid =} \StringTok{"GHCND"}\NormalTok{,}
               \AttributeTok{stationid =} \StringTok{"GHCND:US1ORMT0006"}\NormalTok{,}
               \AttributeTok{startdate =} \StringTok{"2021{-}01{-}01"}\NormalTok{,}
               \AttributeTok{enddate =} \StringTok{"2021{-}02{-}01"}\NormalTok{)}

\NormalTok{precip\_se\_pdx\_2021\_data }\OtherTok{\textless{}{-}}\NormalTok{ precip\_se\_pdx\_2021}\SpecialCharTok{$}\NormalTok{data}

\NormalTok{precip\_se\_pdx\_2021\_data}\SpecialCharTok{$}\NormalTok{date }\OtherTok{\textless{}{-}} \FunctionTok{ymd\_hms}\NormalTok{(precip\_se\_pdx\_2021\_data}\SpecialCharTok{$}\NormalTok{date)}
\CommentTok{\#2022}
\NormalTok{precip\_se\_pdx\_2022 }\OtherTok{\textless{}{-}} \FunctionTok{ncdc}\NormalTok{(}\AttributeTok{datasetid =} \StringTok{"GHCND"}\NormalTok{,}
               \AttributeTok{stationid =} \StringTok{"GHCND:US1ORMT0006"}\NormalTok{,}
               \AttributeTok{startdate =} \StringTok{"2022{-}01{-}01"}\NormalTok{,}
               \AttributeTok{enddate =} \StringTok{"2022{-}02{-}01"}\NormalTok{)}

\NormalTok{precip\_se\_pdx\_2022\_data }\OtherTok{\textless{}{-}}\NormalTok{ precip\_se\_pdx\_2022}\SpecialCharTok{$}\NormalTok{data}

\NormalTok{precip\_se\_pdx\_2022\_data}\SpecialCharTok{$}\NormalTok{date }\OtherTok{\textless{}{-}} \FunctionTok{ymd\_hms}\NormalTok{(precip\_se\_pdx\_2022\_data}\SpecialCharTok{$}\NormalTok{date)}
\CommentTok{\#2023}
\NormalTok{precip\_se\_pdx\_2023 }\OtherTok{\textless{}{-}} \FunctionTok{ncdc}\NormalTok{(}\AttributeTok{datasetid =} \StringTok{"GHCND"}\NormalTok{,}
               \AttributeTok{stationid =} \StringTok{"GHCND:US1ORMT0006"}\NormalTok{,}
               \AttributeTok{startdate =} \StringTok{"2023{-}01{-}01"}\NormalTok{,}
               \AttributeTok{enddate =} \StringTok{"2023{-}02{-}01"}\NormalTok{)}

\NormalTok{precip\_se\_pdx\_2023\_data }\OtherTok{\textless{}{-}}\NormalTok{ precip\_se\_pdx\_2023}\SpecialCharTok{$}\NormalTok{data}

\NormalTok{precip\_se\_pdx\_2023\_data}\SpecialCharTok{$}\NormalTok{date }\OtherTok{\textless{}{-}} \FunctionTok{ymd\_hms}\NormalTok{(precip\_se\_pdx\_2023\_data}\SpecialCharTok{$}\NormalTok{date)}

\NormalTok{last4years }\OtherTok{\textless{}{-}}\NormalTok{ precip\_se\_pdx\_data }\SpecialCharTok{\%\textgreater{}\%} \FunctionTok{full\_join}\NormalTok{(precip\_se\_pdx\_2023\_data)}

\NormalTok{last4years }\OtherTok{\textless{}{-}}\NormalTok{ last4years }\SpecialCharTok{\%\textgreater{}\%} \FunctionTok{full\_join}\NormalTok{(precip\_se\_pdx\_2022\_data)}

\NormalTok{last4years }\OtherTok{\textless{}{-}}\NormalTok{ last4years }\SpecialCharTok{\%\textgreater{}\%} \FunctionTok{full\_join}\NormalTok{(precip\_se\_pdx\_2021\_data)}

\NormalTok{last4years }\OtherTok{\textless{}{-}}\NormalTok{ last4years }\SpecialCharTok{\%\textgreater{}\%} \FunctionTok{full\_join}\NormalTok{(precip\_se\_pdx\_2020\_data)}

\NormalTok{trying }\OtherTok{\textless{}{-}}\NormalTok{ last4years }\SpecialCharTok{\%\textgreater{}\%}
  \FunctionTok{mutate}\NormalTok{(}\AttributeTok{year =} \FunctionTok{year}\NormalTok{(date), }\AttributeTok{newdate =} \FunctionTok{mday}\NormalTok{(date))}

\NormalTok{trying }\SpecialCharTok{\%\textgreater{}\%}
  \FunctionTok{ggplot}\NormalTok{(}\FunctionTok{aes}\NormalTok{(}\AttributeTok{x =}\NormalTok{ newdate, }\AttributeTok{y =}\NormalTok{ value)) }\SpecialCharTok{+}
  \FunctionTok{geom\_col}\NormalTok{() }\SpecialCharTok{+}
  \FunctionTok{facet\_wrap}\NormalTok{(}\SpecialCharTok{\textasciitilde{}}\NormalTok{year, }\AttributeTok{ncol =} \DecValTok{1}\NormalTok{)}
\end{Highlighting}
\end{Shaded}

\includegraphics{lab05_files/figure-latex/unnamed-chunk-9-1.pdf}

\hypertarget{problem-2-from-api-to-r}{%
\subsection{Problem 2: From API to R}\label{problem-2-from-api-to-r}}

For this problem I want you to grab web data by either talking to an API
directly with \texttt{httr} or using an API wrapper. It must be an API
that we have NOT used in class or in Problem 1.

Once you have grabbed the data, do any necessary wrangling to graph it
and/or produce some summary statistics. Draw some conclusions from your
graph and summary statistics.

\hypertarget{api-wrapper-suggestions-for-problem-2}{%
\subsubsection{API Wrapper Suggestions for Problem
2}\label{api-wrapper-suggestions-for-problem-2}}

Here are some potential API wrapper packages. Feel free to use one not
included in this list for Problem 2.

\begin{itemize}
\tightlist
\item
  \texttt{gtrendsR}: ``An interface for retrieving and displaying the
  information returned online by Google Trends is provided. Trends
  (number of hits) over the time as well as geographic representation of
  the results can be displayed.''
\item
  \href{https://github.com/ropensci/rfishbase}{\texttt{rfishbase}}: For
  the fish lovers
\item
  \href{https://github.com/hrbrmstr/darksky}{\texttt{darksky}}: For
  global historical and current weather conditions
\end{itemize}

\begin{Shaded}
\begin{Highlighting}[]
\FunctionTok{library}\NormalTok{(rfishbase)}
\end{Highlighting}
\end{Shaded}

\begin{Shaded}
\begin{Highlighting}[]
\NormalTok{salmon }\OtherTok{\textless{}{-}} \FunctionTok{c}\NormalTok{(}\StringTok{"Oncorhynchus tshawytscha"}\NormalTok{, }\StringTok{"Salmo salar"}\NormalTok{, }\StringTok{"Oncorhynchus keta"}\NormalTok{, }\StringTok{"Oncorhynchus kisutch"}\NormalTok{,  }\StringTok{"Oncorhynchus masou"}\NormalTok{, }\StringTok{"Oncorhynchus gorbuscha"}\NormalTok{, }\StringTok{"Oncorhynchus nerka"}\NormalTok{)}

\NormalTok{selectedfish }\OtherTok{\textless{}{-}} \FunctionTok{fb\_tbl}\NormalTok{(}\StringTok{"species"}\NormalTok{) }\SpecialCharTok{\%\textgreater{}\%} 
  \FunctionTok{mutate}\NormalTok{(}\AttributeTok{sci\_name =} \FunctionTok{paste}\NormalTok{(Genus, Species)) }\SpecialCharTok{\%\textgreater{}\%}
  \FunctionTok{filter}\NormalTok{(sci\_name }\SpecialCharTok{\%in\%}\NormalTok{ salmon)}


\NormalTok{salmonpredators }\OtherTok{\textless{}{-}} \FunctionTok{predators}\NormalTok{(}\AttributeTok{species\_list =}\NormalTok{ salmon)}


\NormalTok{trying2 }\OtherTok{\textless{}{-}}\NormalTok{ salmonpredators }\SpecialCharTok{\%\textgreater{}\%}
  \FunctionTok{group\_by}\NormalTok{(Species) }\SpecialCharTok{\%\textgreater{}\%}
  \FunctionTok{group\_by}\NormalTok{(PredatorI) }\SpecialCharTok{\%\textgreater{}\%}
  \FunctionTok{mutate}\NormalTok{(}\AttributeTok{numberofpredators =} \FunctionTok{n}\NormalTok{())}


\NormalTok{  trying2 }\SpecialCharTok{\%\textgreater{}\%}
  \FunctionTok{ggplot}\NormalTok{(}\FunctionTok{aes}\NormalTok{(}\AttributeTok{x =}\NormalTok{ PredatorI, }\AttributeTok{y =}\NormalTok{ numberofpredators)) }\SpecialCharTok{+}
  \FunctionTok{geom\_col}\NormalTok{() }\SpecialCharTok{+}
  \FunctionTok{facet\_wrap}\NormalTok{(}\SpecialCharTok{\textasciitilde{}}\NormalTok{Species, }\AttributeTok{ncol =} \DecValTok{4}\NormalTok{) }\SpecialCharTok{+} 
    \FunctionTok{theme}\NormalTok{(}\AttributeTok{axis.text.x =} \FunctionTok{element\_text}\NormalTok{(}\AttributeTok{angle =} \DecValTok{90}\NormalTok{, }\AttributeTok{vjust =} \FloatTok{0.5}\NormalTok{, }\AttributeTok{hjust=}\DecValTok{1}\NormalTok{)) }\SpecialCharTok{+}
    \FunctionTok{labs}\NormalTok{(}\AttributeTok{x =} \StringTok{"Primary predator type"}\NormalTok{, }\AttributeTok{y =} \StringTok{"\# of different predator species"}\NormalTok{, }\AttributeTok{title =} \StringTok{"Distribution of predator types across different salmon species"}\NormalTok{)}
\end{Highlighting}
\end{Shaded}

\includegraphics{lab05_files/figure-latex/unnamed-chunk-11-1.pdf}

\hypertarget{problem-3-scraping-reedie-data}{%
\subsection{Problem 3: Scraping Reedie
Data}\label{problem-3-scraping-reedie-data}}

Let's see what lovely data we can pull from Reed's own website.

\begin{enumerate}
\def\labelenumi{\alph{enumi}.}
\tightlist
\item
  Go to \url{https://www.reed.edu/ir/success.html} and scrape the two
  tables.
\end{enumerate}

\begin{Shaded}
\begin{Highlighting}[]
\NormalTok{url }\OtherTok{\textless{}{-}} \StringTok{"https://www.reed.edu/ir/success.html"}

\NormalTok{reedtables }\OtherTok{\textless{}{-}}\NormalTok{ url }\SpecialCharTok{\%\textgreater{}\%}
  \FunctionTok{read\_html}\NormalTok{() }\SpecialCharTok{\%\textgreater{}\%}
  \FunctionTok{html\_nodes}\NormalTok{(}\AttributeTok{css =} \StringTok{"table"}\NormalTok{)}

\NormalTok{employmenttypetable }\OtherTok{\textless{}{-}} \FunctionTok{html\_table}\NormalTok{(reedtables[[}\DecValTok{1}\NormalTok{]], }\AttributeTok{fill =} \ConstantTok{TRUE}\NormalTok{)}


\NormalTok{furtherschoolingtable }\OtherTok{\textless{}{-}} \FunctionTok{html\_table}\NormalTok{(reedtables[[}\DecValTok{2}\NormalTok{]], }\AttributeTok{fill =} \ConstantTok{TRUE}\NormalTok{)}


\NormalTok{fellowshipstable }\OtherTok{\textless{}{-}} \FunctionTok{html\_table}\NormalTok{(reedtables[[}\DecValTok{3}\NormalTok{]], }\AttributeTok{fill =} \ConstantTok{TRUE}\NormalTok{)}
\end{Highlighting}
\end{Shaded}

\begin{enumerate}
\def\labelenumi{\alph{enumi}.}
\setcounter{enumi}{1}
\tightlist
\item
  Grab and print out the table that is entitled ``GRADUATE SCHOOLS MOST
  FREQUENTLY ATTENDED BY REED ALUMNI''. Why is this data frame not in a
  tidy format?
\end{enumerate}

\begin{Shaded}
\begin{Highlighting}[]
\FunctionTok{print}\NormalTok{(furtherschoolingtable)}
\end{Highlighting}
\end{Shaded}

\begin{verbatim}
## # A tibble: 11 x 4
##    MBAs               JDs                       PhDs                     MDs    
##    <chr>              <chr>                     <chr>                    <chr>  
##  1 U. of Chicago      Lewis & Clark  Law School U.C., Berkeley           Oregon~
##  2 Portland State U.  U.C., Berkeley            U. of Washington         U. of ~
##  3 Harvard U.         U. of Oregon              U. of Chicago            Washin~
##  4 U. of Washington   U. of Washington          Stanford U.              UC., S~
##  5 Columbia U.        New York U.               U. of Oregon             Stanfo~
##  6 U of Pennsylvania. U. of Chicago             Harvard U.               Harvar~
##  7 Stanford U.        Yale U.                   Cornell U.               Case W~
##  8 Yale U.            Harvard U.                Columbia U.              Cornel~
##  9 U.C., Berkeley     U.C. Hastings Law School  U.C., Los Angeles        Johns ~
## 10 U. of Oregon       Cornell U.                Yale U.                  U. of ~
## 11 UC., Los Angeles.  Georgetown U.             U. of Wisconsin, Madison U. of ~
\end{verbatim}

The different schools are separated into columns by type of graduate
degree. In a tidy data format, this data frame would have two columns:
one that's a ``school name'' and one thats a ``degree type''.

\begin{enumerate}
\def\labelenumi{\alph{enumi}.}
\setcounter{enumi}{2}
\tightlist
\item
  Wrangle the data into a tidy format. Glimpse the resulting data frame.
\end{enumerate}

\begin{Shaded}
\begin{Highlighting}[]
\NormalTok{  furtherschoolingtabletidy }\OtherTok{\textless{}{-}} \FunctionTok{pivot\_longer}\NormalTok{(furtherschoolingtable, }
                          \AttributeTok{cols =} \FunctionTok{c}\NormalTok{(MBAs, JDs, PhDs, MDs), }
                          \AttributeTok{names\_to =} \StringTok{"degree\_type"}\NormalTok{, }
                          \AttributeTok{values\_to =} \StringTok{"school\_name"}\NormalTok{)}


\FunctionTok{glimpse}\NormalTok{(furtherschoolingtabletidy)}
\end{Highlighting}
\end{Shaded}

\begin{verbatim}
## Rows: 44
## Columns: 2
## $ degree_type <chr> "MBAs", "JDs", "PhDs", "MDs", "MBAs", "JDs", "PhDs", "MDs"~
## $ school_name <chr> "U. of Chicago", "Lewis & Clark  Law School", "U.C., Berke~
\end{verbatim}

\begin{enumerate}
\def\labelenumi{\alph{enumi}.}
\setcounter{enumi}{3}
\tightlist
\item
  Now grab the ``OCCUPATIONAL DISTRIBUTION OF ALUMNI'' table and turn it
  into an appropriate graph. What conclusions can we draw from the
  graph?
\end{enumerate}

\begin{Shaded}
\begin{Highlighting}[]
\CommentTok{\# Hint: Use \textasciigrave{}parse\_number()\textasciigrave{} within \textasciigrave{}mutate()\textasciigrave{} to fix one of the columns}
\NormalTok{employmenttypetable }\OtherTok{\textless{}{-}}\NormalTok{ employmenttypetable }\SpecialCharTok{\%\textgreater{}\%}
\FunctionTok{rename}\NormalTok{(}\AttributeTok{employmenttype =}\NormalTok{ X1) }\SpecialCharTok{\%\textgreater{}\%}
\FunctionTok{rename}\NormalTok{(}\AttributeTok{propfalse =}\NormalTok{ X2) }\SpecialCharTok{\%\textgreater{}\%}
  \FunctionTok{mutate}\NormalTok{(}\AttributeTok{prop =} \FunctionTok{as.numeric}\NormalTok{(}\FunctionTok{sub}\NormalTok{(}\StringTok{"\%"}\NormalTok{, }\StringTok{""}\NormalTok{, propfalse)))}

\FunctionTok{ggplot}\NormalTok{(}\AttributeTok{data =}\NormalTok{ employmenttypetable, }\FunctionTok{aes}\NormalTok{(}\AttributeTok{x =}\NormalTok{ employmenttype, }\AttributeTok{y =}\NormalTok{ prop)) }\SpecialCharTok{+}
  \FunctionTok{geom\_col}\NormalTok{() }\SpecialCharTok{+}
  \FunctionTok{theme}\NormalTok{(}\AttributeTok{axis.text.x =} \FunctionTok{element\_text}\NormalTok{(}\AttributeTok{angle =} \DecValTok{90}\NormalTok{, }\AttributeTok{vjust =} \FloatTok{0.5}\NormalTok{, }\AttributeTok{hjust=}\DecValTok{1}\NormalTok{)) }\SpecialCharTok{+}
  \FunctionTok{labs}\NormalTok{(}\AttributeTok{x =} \StringTok{"Area of Employment"}\NormalTok{, }\AttributeTok{y =} \StringTok{"Proportion of Reed Graduates"}\NormalTok{)}
\end{Highlighting}
\end{Shaded}

\includegraphics{lab05_files/figure-latex/unnamed-chunk-15-1.pdf}

\begin{enumerate}
\def\labelenumi{\alph{enumi}.}
\setcounter{enumi}{4}
\tightlist
\item
  Let's now grab the Reed graduation rates over time. Grab the data from
  \href{https://www.reed.edu/ir/gradrateshist.html}{here}.
\end{enumerate}

\begin{Shaded}
\begin{Highlighting}[]
\NormalTok{url2 }\OtherTok{\textless{}{-}} \StringTok{"https://www.reed.edu/ir/gradrateshist.html"}
  
  
\NormalTok{  gradtablehtml }\OtherTok{\textless{}{-}}\NormalTok{ url2 }\SpecialCharTok{\%\textgreater{}\%}
  \FunctionTok{read\_html}\NormalTok{() }\SpecialCharTok{\%\textgreater{}\%}
  \FunctionTok{html\_nodes}\NormalTok{(}\AttributeTok{css =} \StringTok{"table"}\NormalTok{)}

\NormalTok{gradtable }\OtherTok{\textless{}{-}} \FunctionTok{html\_table}\NormalTok{(gradtablehtml[[}\DecValTok{1}\NormalTok{]], }\AttributeTok{fill =} \ConstantTok{TRUE}\NormalTok{)}
\end{Highlighting}
\end{Shaded}

Do the following to clean up the data:

\begin{itemize}
\tightlist
\item
  Rename the column names.
\end{itemize}

\begin{Shaded}
\begin{Highlighting}[]
\CommentTok{\# Hint}
\FunctionTok{colnames}\NormalTok{(gradtable) }\OtherTok{\textless{}{-}} \FunctionTok{c}\NormalTok{(}\StringTok{"StartYear"}\NormalTok{, }\StringTok{"ClassSize"}\NormalTok{, }\StringTok{"Fouryeargradrate"}\NormalTok{, }\StringTok{"Fiveyeargradrate"}\NormalTok{, }\StringTok{"Sixyeargradrate"}\NormalTok{)}
\end{Highlighting}
\end{Shaded}

\begin{itemize}
\tightlist
\item
  Remove any extraneous rows.
\end{itemize}

\begin{Shaded}
\begin{Highlighting}[]
\CommentTok{\# Hint}
\NormalTok{fixinggradtable }\OtherTok{\textless{}{-}}\NormalTok{ gradtable }\SpecialCharTok{\%\textgreater{}\%}
\FunctionTok{filter}\NormalTok{(StartYear }\SpecialCharTok{!=} \StringTok{"First{-}year students who entered fall of..."}\NormalTok{) }\SpecialCharTok{\%\textgreater{}\%}
  \FunctionTok{filter}\NormalTok{(StartYear }\SpecialCharTok{!=} \StringTok{"2019"}\NormalTok{) }\SpecialCharTok{\%\textgreater{}\%}
  \FunctionTok{filter}\NormalTok{(StartYear }\SpecialCharTok{!=} \StringTok{"2018"}\NormalTok{) }\SpecialCharTok{\%\textgreater{}\%}
  \FunctionTok{filter}\NormalTok{(StartYear }\SpecialCharTok{!=} \StringTok{"2017"}\NormalTok{)}
\end{Highlighting}
\end{Shaded}

\begin{itemize}
\tightlist
\item
  Reshape the data so that there are columns for

  \begin{itemize}
  \tightlist
  \item
    Entering class year
  \item
    Cohort size
  \item
    Years to graduation
  \item
    Graduation rate
  \end{itemize}
\item
  Make sure each column has the correct class.
\end{itemize}

\begin{Shaded}
\begin{Highlighting}[]
\NormalTok{fixinggradtable2 }\OtherTok{\textless{}{-}}\NormalTok{ fixinggradtable }\SpecialCharTok{\%\textgreater{}\%}
  \FunctionTok{mutate}\NormalTok{(}\AttributeTok{Fouryearprop =} \FunctionTok{as.numeric}\NormalTok{(}\FunctionTok{sub}\NormalTok{(}\StringTok{"\%"}\NormalTok{, }\StringTok{""}\NormalTok{, Fouryeargradrate)), }
         \AttributeTok{Fiveyearprop =} \FunctionTok{as.numeric}\NormalTok{(}\FunctionTok{sub}\NormalTok{(}\StringTok{"\%"}\NormalTok{, }\StringTok{""}\NormalTok{, Fiveyeargradrate)),}
         \AttributeTok{Sixyearprop =} \FunctionTok{as.numeric}\NormalTok{(}\FunctionTok{sub}\NormalTok{(}\StringTok{"\%"}\NormalTok{, }\StringTok{""}\NormalTok{, Sixyeargradrate)))}
\end{Highlighting}
\end{Shaded}

\begin{enumerate}
\def\labelenumi{\alph{enumi}.}
\setcounter{enumi}{5}
\tightlist
\item
  Create a graph comparing the graduation rates over time and draw some
  conclusions.
\end{enumerate}

\begin{Shaded}
\begin{Highlighting}[]
\NormalTok{fixinggradtable2 }\SpecialCharTok{\%\textgreater{}\%}
  \FunctionTok{ggplot}\NormalTok{(}\FunctionTok{aes}\NormalTok{(}\AttributeTok{x =}\NormalTok{ StartYear, }\AttributeTok{y =}\NormalTok{ Fouryearprop)) }\SpecialCharTok{+}
  \FunctionTok{geom\_col}\NormalTok{() }\SpecialCharTok{+}
  \FunctionTok{theme}\NormalTok{(}\AttributeTok{axis.text.x =} \FunctionTok{element\_text}\NormalTok{(}\AttributeTok{angle =} \DecValTok{90}\NormalTok{, }\AttributeTok{vjust =} \FloatTok{0.5}\NormalTok{, }\AttributeTok{hjust=}\DecValTok{1}\NormalTok{)) }\SpecialCharTok{+}
  \FunctionTok{labs}\NormalTok{(}\AttributeTok{x =} \StringTok{"Freshman class year"}\NormalTok{, }\AttributeTok{y =} \StringTok{"Proportion of students who graduate in four years"}\NormalTok{, }\AttributeTok{title =} \StringTok{"Four{-}year graduation rates at Reed College"}\NormalTok{) }
\end{Highlighting}
\end{Shaded}

\includegraphics{lab05_files/figure-latex/unnamed-chunk-20-1.pdf}

\begin{Shaded}
\begin{Highlighting}[]
\NormalTok{fixinggradtable2 }\SpecialCharTok{\%\textgreater{}\%}
  \FunctionTok{ggplot}\NormalTok{(}\FunctionTok{aes}\NormalTok{(}\AttributeTok{x =}\NormalTok{ StartYear, }\AttributeTok{y =}\NormalTok{ Fiveyearprop)) }\SpecialCharTok{+}
  \FunctionTok{geom\_col}\NormalTok{() }\SpecialCharTok{+}
  \FunctionTok{theme}\NormalTok{(}\AttributeTok{axis.text.x =} \FunctionTok{element\_text}\NormalTok{(}\AttributeTok{angle =} \DecValTok{90}\NormalTok{, }\AttributeTok{vjust =} \FloatTok{0.5}\NormalTok{, }\AttributeTok{hjust=}\DecValTok{1}\NormalTok{)) }\SpecialCharTok{+}
  \FunctionTok{labs}\NormalTok{(}\AttributeTok{x =} \StringTok{"Freshman class year"}\NormalTok{, }\AttributeTok{y =} \StringTok{"Proportion of students who graduate in five years"}\NormalTok{, }\AttributeTok{title =} \StringTok{"Five{-}year graduation rates at Reed College"}\NormalTok{) }
\end{Highlighting}
\end{Shaded}

\includegraphics{lab05_files/figure-latex/unnamed-chunk-20-2.pdf}

\begin{Shaded}
\begin{Highlighting}[]
\NormalTok{fixinggradtable2 }\SpecialCharTok{\%\textgreater{}\%}
  \FunctionTok{ggplot}\NormalTok{(}\FunctionTok{aes}\NormalTok{(}\AttributeTok{x =}\NormalTok{ StartYear, }\AttributeTok{y =}\NormalTok{ Sixyearprop)) }\SpecialCharTok{+}
  \FunctionTok{geom\_col}\NormalTok{() }\SpecialCharTok{+}
  \FunctionTok{theme}\NormalTok{(}\AttributeTok{axis.text.x =} \FunctionTok{element\_text}\NormalTok{(}\AttributeTok{angle =} \DecValTok{90}\NormalTok{, }\AttributeTok{vjust =} \FloatTok{0.5}\NormalTok{, }\AttributeTok{hjust=}\DecValTok{1}\NormalTok{)) }\SpecialCharTok{+}
  \FunctionTok{labs}\NormalTok{(}\AttributeTok{x =} \StringTok{"Freshman class year"}\NormalTok{, }\AttributeTok{y =} \StringTok{"Proportion of students who graduate in six years"}\NormalTok{, }\AttributeTok{title =} \StringTok{"Six{-}year graduation rates at Reed College"}\NormalTok{) }
\end{Highlighting}
\end{Shaded}

\includegraphics{lab05_files/figure-latex/unnamed-chunk-20-3.pdf}

Four-year graduation rates are going up quite a bit over time. Since
these proportions are cumulative, we can see in the six-year graduation
rates that total proportions of students graduating has increased
slightly from \textasciitilde60\% of students in the 1980s to
\textasciitilde80\% of students in the 2010s. Alongside this increase in
total amount of graduating students is a decrease between four-year
graduation rates and six-year graduation rates, which tells us that
recently, more students are graduating in four years and are not taking
additional time to graduate.

\end{document}
